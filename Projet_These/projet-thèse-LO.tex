%\documentclass[twoside,french,11pt]{report}
%\documentclass[twoside,french,11pt]{msu-thesis}

\documentclass[twoside,french,11pt]{amsart}

%amsart

%\usepackage[latin1]{inputenc}
%\usepackage[french]{babel}
%\usepackage{colortbl}
%\usepackage{graphicx}

\usepackage[hmargin=2.5cm, vmargin=2cm]{geometry}
\usepackage[T1]{fontenc}
\usepackage[utf8]{inputenc}

\pagestyle{plain}

\usepackage{babel}
%\usepackage[french]{babel}
\usepackage{amsfonts}
\usepackage{amsthm}
\usepackage{amsmath}
\usepackage{amssymb}
%\usepackage[latin1]{inputenc}
\usepackage[all]{xy}

\usepackage[colorlinks=true]{hyperref}

% Generated with LaTeXDraw 2.0.8
% Tue Oct 09 07:11:11 ACT 2012
\usepackage[usenames,dvipsnames]{pstricks}
\usepackage{epsfig}
\usepackage{pst-grad} % For gradients
\usepackage{pst-plot} % For axes
\usepackage{graphicx}

\setcounter{secnumdepth}{4}
\setcounter{tocdepth}{4}

\linespread{1.3}

\date{}

\newtheorem{definition}{D\'efinition}[section]
\newtheorem{defnot}{D\'efinitions et Notations}[section]
\newtheorem{remarque}{Remarque}[section]
\newtheorem{proposition}{Proposition}[section]
\newtheorem{exemple}{Exemples}[section]
\newtheorem{lemme}{Lemme}[section]
\newtheorem{theorem}{Th\'eor\`eme}[section]
\newtheorem{corollary}{Corollaire}[section]
\newtheorem{propriete}{Propri\'et\'e}[section]
\newtheorem{algo}{Algorithme}[section]
\usepackage{graphicx}

%\title{\Large{  THESE DE TROISIEME CYCLE  ÈS SCIENCES MATHÉMATIQUES  \\
         %OPTION: \underline{CRYPTOLOGIE}}}

\begin{document}
%\tableofcontents
%\maketitle
\begin{titlepage}
	\addtolength{\hoffset}{-1cm}
	\centering
	{\scshape\large RÉPUBLIQUE DU SÉNÉGAL \\MINISTÈRE DE L'ENSEIGNEMENT SUPÉRIEUR,\\
		DE LA RECHERCHE ET DE L'INNOVATION \par}
	\vspace{1cm}
	\includegraphics[width=0.30\textwidth]{logo}\par\vspace{0.2cm}
	{\scshape\Large Université Gaston Berger de Saint-Louis \par}
	\vspace{0.5cm}
	{\scshape\Large ÉCOLE DOCTORALE DES SCIENCES ET DES TECHNOLOGIES \par}
	\vspace{1.5cm}
	{\scshape Projet de Thèse \par}
	
	\vspace{0.5cm}
	
	
	{\LARGE\bfseries \textbf{Chiffrement unidirectionnel et chiffrement bidirectionnel: constructions, modèle de sécurité et applications} \par}
	\vspace{2.5cm}

	{\LARGE\itshape Mamadou Makhtar \textsc{LO} \par}
	{\large Email: Lo.mamadou-makhtar@ugb.edu.sn \par}
	\vfill
	
	% Bottom of the page
	{\large \textbf{Année académique 2020-2021}\par}
\end{titlepage}


\newpage

\tableofcontents

\newpage

\section*{Contexte}

Un des objectifs de base de la cryptographie en général et du chiffrement en particulier est d'assurer la confidentialité de données. Il paraît même nécessaire qu'un système de chiffrement ne peut permettre l'accès au message (texte clair) qu'au destinataire du texte chiffré. Ainsi, transférer le texte clair à un autre destinataire nécessiterait un accès au texte clair et à la clé de chiffrement de l'autre destinataire.\\
Cependant avec le développement des outils technologiques tels que le stockage sur réseau (Drive, iCloud, Mega etc..), le Cloud Computing ou le transfert de fichiers entre objets connectés (IoT), il devient nécessaire de permettre le partage de données et le contrôle d'accès des utilisateurs du réseau tout en assurant la sécurité des données. En 1998, Matt Blaze et al. \cite{Blaze98} introduisaient la notion d' \textit{atomic proxy function} qui permettrait de transformer un texte chiffré pour une clé particulière en un texte chiffré pour une autre clé sans pour autant avoir accès au texte clair ou aux clés de déchiffrement. On parle de serveur de re-chiffrement ou \textit{Proxy Re-Encryption}(PRE).\\

Les PRE qui étaient à l'origine conçus pour être des systèmes de suivi de courriers chiffrés sont aujourd'hui utilisés pour le contrôle d'accès et les partages de données notamment en \textit{Cloud Computing}. \cite{devigne:tel-01081377} \cite{ateniese2006improved}
\cite{liu2020development}. \\
Dans leurs travaux, Blaze and al. \cite{Blaze98} proposèrent un schéma de PRE basé sur El Gamal \cite{elgamal1985public}. Toutefois, il aura fallu attendre Ateniese et Al. \cite{ateniese2006improved} pour avoir une première définition formelle des PRE et de leurs modèles de sécurité. On a ainsi pu observer une évolution des PRE ces vingt dernières années avec la construction de nouveaux modèles, des preuves de sécurité et une utilisation plus large.
\section{Chiffrement Unidirectionnel et Chiffrement Bidirectionnel}

Un schéma de serveur de re-chiffrement peut être défini par la donnée de cinq algorithmes \{\textit{Clé}, \textit{Délégation}, \textit{Chiffrer}, \textit{Re-Chiffrer}, \textit{Dechiffrer}\} tels que:
\begin{itemize}
	\item \textit{Clé} génère les clés de chiffrement et déchiffrement.
	\item \textit{Délégation} génère la clé de délégation ou \textit{clé proxy}.
	\item \textit{Chiffrer} chiffre le message à l'aide de la clé de chiffrement.
	\item \textit{Re-Chiffrer} transforme (re-chiffre) à l'aide de la \textit{clé proxy} un texte chiffré pour le délégant en un texte chiffré pour le délégué.
	\item  \textit{Dechiffrer} déchiffre le texte chiffré à l'aide de la clé de déchiffrement.
\end{itemize}
Ainsi, suivant la confiance entre le délégant(\textit{Alice}) et le délégué (\textit{Bob}), on peut distinguer deux catégories de PRE: celle basée sur le \textbf{\textit{chiffrement bidirectionnel}} et celle sur le \textbf{\textit{chiffrement unidirectionnel}}.

\subsection{Chiffrement Unidirectionnel}
Pour un schéma de serveur de re-chiffrement unidirectionnel, la délégation se fait uniquement dans un sens. La \textit{clé proxy} ne permet que de transformer un chiffré pour \textit{Alice} en un chiffré pour \textit{Bob}. On doit utiliser une autre \textit{clé proxy} pour avoir la délégation inverse.  \\ Ivan et Al. \cite{ivan2003proxy} seront les premiers en 2003 à construire une méthode générique pour construire un schéma pouvant être considéré comme unidirectionnel.

\subsection{Chiffrement Bidirectionnel}
Dans le cas du chiffrement bidirectionnel, la \textit{clé proxy} utilisée pour transformer un chiffré pour \textit{Alice} en un chiffré pour \textit{Bob} peut aussi être utilisée pour transformer un texte chiffré pour \textit{Bob} en un texte pour \textit{Alice}. La délégation se fait dans les deux sens.\\
Le schéma proposé par Blazer en 1998 peut être considéré comme bidirectionnel.

\section{Problématique et Objectifs}
\subsection{Problématique}
Le modèle proposé par Blaze et Al.\cite{blaze98} en 1998  ne résistait pas aux attaques par collusion, le proxy et le délégué pouvait aussi toujours collaborer pour retrouver la clé du délégant. En 2003, Ivan et Al.\cite{ivan2003proxy} proposeront une méthode générique pour construire un serveur de re-chiffrement unidirectionnel, toutefois ces schémas ne résistaient toujours pas à l'attaque par collusion et n'optimiser pas l'utilisation des clés.\\
Cependant, en 2005, Ateniese et Al. \cite{ateniese2006improved} formalisent la notion de PRE et ses modèles de sécurité et construisent un schéma PRE résistant aux attaques par collusion et achevant une sécurité CPA. Ils posèrent le problème de la construction de modèle achevant la sécurité CCA. Plusieurs modèles ont pu ainsi être construits: Canetti et Al. avec un PRE bidirectionnel et CCA-sûr (2007)\cite{canetti2007chosen}, Libert et Al. avec un PRE unidirectionnel RCCA-sûr (2008) \cite{libert2008unidirectional}, Shao et Al. avec le premier modèle CCA-sûr sans forme bilinéaire(2009)\cite{shao2009cca}, Kirshanova et Al avec leur modèle basé sur LWE et CCA-sûr(2014)\cite{kirshanova2014proxy}.\\
L'étude de schémas de re-chiffrement sûrs et efficaces et la construction de modèles applicables contribuerait au développement de la recherche et la mise en ?uvre des serveurs de chiffrement.  

 \subsection{Objectifs Scientifiques}
De manière concrète, nous nous intéresserons dans ce travail aux aspects théoriques et pratiques des serveurs de re-chiffrement unidirectionnel et de re-chiffrement bidirectionnel. Nous travaillerons, en particulier, sur les modèles de sécurité et la construction de schémas efficaces et leurs applications.

\subsection{Méthodologie de recherche}
La recherche couvrira tous les aspects théoriques et pratiques de la question, notamment :
\begin{itemize}
	\item les aspects définitionnels, par l'analyse des modèles existants et la proposition éventuelle de nouveaux modèles ;
	\item les aspects cryptographiques théoriques par l'analyse des modèles de sécurité et le design de ;
	\item les aspects cryptographiques pratiques par l'implantation et le déploiement de schémas à sécurité prouvée dans les modèles proposées.
\end{itemize}
\newpage
\LARGE \textbf{Équipe du projet:}\\
\Large \textbf{Directeur de thèse :} \textit{Pr. Mohammed Ben \textsc{Maaouia}}, Maître de Conférences,
Directeur du Labo ACCA;\\
\Large \textbf{Porteur du projet de thèse :} \textit{Dr Demba \textsc{SOW}}, Maître–assistant\\



\vspace{2cm}
\bibliographystyle{plain}
\bibliography{theseLO}

\end{document} 