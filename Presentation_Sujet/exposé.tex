\documentclass{beamer}
\usetheme{default}

\title{Chiffrement Unidirectionnel et Chiffrement Bidirectionnel}
\subtitle{Constructions, Modèle de Sécurité et Applications}
\author{Mamadou Makhtar LO}
\begin{document}
\begin{frame}[plain]
    \maketitle
\end{frame}
\section*{Introduction}

\begin{frame}{Introduction}
	\begin{block}{Contexte}
	\begin{itemize}
		\item Développements Technologiques: Cloud Storage, Cloud Computing, IoT etc..
		\item Partage de fichiers entre utilisateurs
		\item Besoins en Sécurité: Délégation, Contrôle d'accès..\\
		Comment partager un fichier sans divulguer le message ou les clé de déchiffrements ? 
	\end{itemize}
\end{block}
\end{frame}

\begin{frame}{Introduction}
	\begin{block}{Objet:}
		Serveur de re-chiffrement (ou Proxy Re-Encryption)
		\begin{itemize}
			\item Transforme un texte chiffré pour une clé particulière en un texte chiffré pour une autre clé:\\
			sans avoir accès au texte clair,\\
			sans avoir accès aux clés de déchiffrement.
			\item Selon le sens de la délégation:
			Chiffrement Unidirectionnel\\
			Chiffrement Bidirectionnel.
		\end{itemize}
		
	\end{block}
\end{frame}

\begin{frame}{Introduction}
	\begin{block}{Problématique}
		\begin{itemize}
			\item Construction de serveurs de re-chiffrement (unidirectionnel/Bidirectionnel)
			\item Preuves de Sécurité
			\item Applicabilité et efficacité
		\end{itemize}
	\end{block}
\end{frame}


\begin{frame}{Introduction}
	\begin{block}{Objectifs}
		\begin{itemize}
			\item Aspects définitionnels de la notion de PRE (définitions, particularités, modèles existants ...)
			\item Aspects cryptographiques théoriques (Proposition de modèles, analyse de la sécurité, applications...)
			\item Aspects cryptographiques pratiques (Implémentation, tests d'applicabilité...)
		\end{itemize}
	\end{block}
\end{frame}

\section{Chiffrement Unidirectionnel et Chiffrement Bidirectionnel}
\subsection{Serveur de re-chiffrement}
\begin{frame}{Serveur de re-chiffrement (PRE)}
	\begin{block}{Proxy Re-Encryption}
		\begin{itemize}
			\item Proposé en 1998 par Matt Blaze, Gerrit Bleumer et Martin Strauss
			\item "Atomic Proxy function"
			\item Développement (Méthode générique 2003, Formalisation en 2005 etc.. ) 
		\end{itemize}
	\end{block}
	\begin{block}{Principe de fonctionnement}
	PRE:  \{\textit{Clé}, \textit{Délégation}, \textit{Chiffrer}, \textit{Re-Chiffrer}, \textit{Dechiffrer}\} tels que:
	\begin{itemize}
		\item \textit{Clé} génère les clés de chiffrement et déchiffrement.
		\item \textit{Délégation} génère la clé de délégation ou \textit{clé proxy}.
		\item \textit{Chiffrer} chiffre le message à l'aide de la clé de chiffrement.
		\item \textit{Re-Chiffrer} transforme (re-chiffre) à l'aide de la \textit{clé proxy} un texte chiffré pour A en un texte chiffré pour B.
		\item  \textit{Dechiffrer} déchiffre le texte chiffré à l'aide de la clé de déchiffrement.
	\end{itemize}
	\end{block}
\end{frame}

\subsection{Chiffrement Unidirectionnel}
\begin{frame}{Chiffrement Unidirectionnel}
	
\end{frame}

\subsection{Chiffrement Bidirectionnel}
\begin{frame}{Chiffrement Bidirectionnel}
	
\end{frame}

\section{PRE basé sur le modèle Generalized Elgamal}
\subsection{Le schéma Generalized Elgamal}
\begin{frame}{Le schéma "lite" Cramer-Shoup}
	
\end{frame}

\subsection{Le schéma "lite" Cramer-Shoup}
\begin{frame}{Le schéma "lite" Cramer-Shoup}
	
\end{frame}

\subsection{Construction de Serveur de Re-chiffrement}
\begin{frame}{Chiffrement Bidirectionnel}
	
\end{frame}




\begin{frame}{Conclusion}
\end{frame}
\end{document}
