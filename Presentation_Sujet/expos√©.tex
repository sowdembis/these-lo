\documentclass{beamer}
\usetheme{default}
\setbeamertemplate{footline} % Only page number at the bottom
{\begin{minipage}{125mm} \vspace{-3 mm} \hfill \insertframenumber \end{minipage}}

\title{Chiffrement Unidirectionnel et Chiffrement Bidirectionnel}
\subtitle{Constructions, Modèle de Sécurité et Applications}
\author{Mamadou Makhtar LO}
\begin{document}
\begin{frame}[plain]
    \maketitle
\end{frame}
\begin{frame}[plain]
	\tableofcontents
\end{frame}
\section{Introduction}
\subsection{Contexte}

\begin{frame}{Introduction}
	\begin{block}{Contexte}
	\begin{itemize}
		\item Développements Technologiques: Cloud Storage, Cloud Computing, IoT etc.. \pause
		\item Partage de fichiers entre utilisateurs \pause
		\item Besoins en Sécurité: Délégation, Contrôle d'accès..\\ \pause
		Comment partager un fichier sans divulguer le message ou les clé de déchiffrements ? 
	\end{itemize}
\end{block}
\end{frame}

\begin{frame}{Introduction}
	\begin{block}{Objet:}
		Serveur de re-chiffrement (ou Proxy Re-Encryption)\pause
		\begin{itemize}
			\item Transforme un texte chiffré pour une clé particulière en un texte chiffré pour une autre clé:\\ \pause
			sans avoir accès au texte clair,\\ \pause
			sans avoir accès aux clés de déchiffrement. \pause
			\item Selon le sens de la délégation: \pause
			Chiffrement Unidirectionnel\\ \pause
			Chiffrement Bidirectionnel. \pause
		\end{itemize}
		
	\end{block}
\end{frame}
\subsection{Problématique}
\begin{frame}{Introduction}
	\begin{block}{Problématique}
		\begin{itemize}
			\item Construction de serveurs de re-chiffrement (unidirectionnel/Bidirectionnel)\pause
			\item Preuves de Sécurité \pause
			\item Applicabilité et efficacité \pause
		\end{itemize}
	\end{block}
\end{frame}

\subsection{Objectifs}
\begin{frame}{Introduction}
	\begin{block}{Objectifs}
		\begin{itemize}
			\item Aspects définitionnels de la notion de PRE (définitions, particularités, modèles existants ...) \pause
			\item Aspects cryptographiques théoriques (Proposition de modèles, analyse de la sécurité, applications...) \pause
			\item Aspects cryptographiques pratiques (Implémentation, tests d'applicabilité...) \pause
		\end{itemize}
	\end{block}
\end{frame}

\section{Chiffrement Unidirectionnel et Chiffrement Bidirectionnel}
\subsection{Serveur de re-chiffrement}
\begin{frame}{Serveur de re-chiffrement (PRE)}
	\begin{block}{Proxy Re-Encryption}
		\begin{itemize}
			\item Proposé en 1998 par Blaze, Bleumer et Strauss \pause
			\item "Atomic Proxy function" \pause
			\item Développement (Méthode générique 2003 (), Formalisation en 2005 (Ateniese) etc.. ) \pause 
		\end{itemize}
	\end{block}
	\begin{block}{Principe de fonctionnement}
	PRE:  \{\textit{Clé}, \textit{Délégation}, \textit{Chiffrer}, \textit{Re-Chiffrer}, \textit{Dechiffrer}\} tels que: \pause
	\begin{itemize}
		\item \textit{Clé} génère les clés de chiffrement et déchiffrement. \pause
		\item \textit{Délégation} génère la clé de délégation ou \textit{clé proxy}. \pause
		\item \textit{Chiffrer} chiffre le message à l'aide de la clé de chiffrement. \pause
		\item \textit{Re-Chiffrer} transforme (re-chiffre) à l'aide de la \textit{clé proxy} un texte chiffré pour A en un texte chiffré pour B. \pause
		\item  \textit{Dechiffrer} déchiffre le texte chiffré à l'aide de la clé de déchiffrement. \pause
	\end{itemize}
	\end{block}
\end{frame}


\subsection{Chiffrement Unidirectionnel}
\begin{frame}{Chiffrement Unidirectionnel}
	
	\begin{itemize}
		\item A peut déléguer à B mais pas inversement. \pause
		\item Premier modèle avec Alan en 2003. \pause
		\item Confiance de A en B. \pause
		\item Formalisation:\\ \pause
		Soit $E$ l'événement "la clé $\pi_{A\to B}$ est calculable en temps polynomial à partir de $\pi_{B\to A}$"\\ \pause
		$P(E)$ est à distance négligeable de 0. \pause
		\item Délégation "pure". \pause
	\end{itemize}
	
\end{frame}



\subsection{Chiffrement Bidirectionnel}

\begin{frame}{Chiffrement Bidirectionnel}
		\begin{itemize}
		\item A peut déléguer à B et inversement. \pause
		\item Premier modèle avec Blaze et Al. en 1998. \pause
		\item Confiance mutuelle entre A et B. \pause
		\item Formalisation:\\ \pause
		Soit $E$ l'événement "la clé $\pi_{A\to B}$ est calculable en temps polynomial à partir de $\pi_{B\to A}$"\\ \pause
		$P(E)$ est à distance négligeable de 1. \pause
		\item Clé proxy souvent de la forme $x_A/x_B$ ou $x_A-x_B$ \pause
		\item Partage d'accès, répertoire de travail... \pause
		\end{itemize}
\end{frame}


\section{PRE basé sur le modèle Generalized Elgamal}
\subsection{Le schéma Generalized Elgamal}
\begin{frame}{Le schéma Generalized Elgamal}
	\begin{block}{Algorithme de génération de clé}
		\begin{itemize}
			\item Choisir un groupe cyclique $G$ d'ordre suffisamment large $d$ tel que $G=\langle g \rangle$. \pause
			
			\item Choisir aléatoirement deux entiers $r$ et $k$ suffisamment large avec $ 2 < k < d$ et $r$ et calculer $kd$. \pause
			\item Avec l'algorithme de la division euclidienne, calculer $(s,t)$ tel que $kd= rs+t$ où $t = kd \mod s$. \pause
			
			\item Calculer $\gamma = g^{s} $ et $\delta = g^{t} \in G$; avec $\gamma \neq 1$ et $\delta \neq 1$. \pause
		\end{itemize}
		La clé publique est $((\gamma, \delta), G)$ et la clé privée $( r, G)$. \pause
		
	\end{block}
\end{frame}
\begin{frame}{Le schéma Generalized Elgamal}
		\begin{block}{Algorithme de Chiffrement}
		    Pour chifrrer un message $m$ avec $((\gamma, \delta), d, 
		G)$: \pause
		\begin{itemize}
			\item Choisir aléatoirement un entier $2<\alpha< d=\#G$ tel que $\alpha$ 
			et $\#G$ premiers entre eux. \pause
			\item Calculer $ c_{1}=\gamma^{\alpha} $ et
			$\lambda=\delta^{\alpha}\in G$ avec $ c_{1}\neq 1$ et $
			\lambda \neq 1$. \pause
			\item Transformer $m$ en un élément de $G$ et calculer
			$c_{2}=\lambda m$ in $G$. \pause
		\end{itemize}
		
		Le texte chiffré est $(c_{1}, c_{2})$. \pause
		\end{block}
			\begin{block}{Algorithme de Dechiffrement.}
			Pour déchiffrer on a juste besoin de calculer $c_{1}^{r}c_{2}$. \pause
			
		\end{block}
\end{frame}


\subsection{Condtruction de Serveur de Rechiffrement}
\begin{frame}{PRE basé sur ElGamal Généralisé}
	\begin{itemize}
		\item Étude de la sécurité du schéma \pause
		\item Construction des algorithmes de Délégation et de Re-chiffrement \pause
		\item Preuves de Sécurité \pause
		\item Applications: Implémentation, test etc..  \pause
	\end{itemize}
	
\end{frame}

\section{Conclusion}
\begin{frame}{Conclusion}
	
	\begin{itemize}
		\item Utilité des Serveurs de Re-chiffrement \pause
		\item Développement des PRE et problématiques \pause
		\item Objectifs et Contributions \pause
		\item Méthodologie de recherche \pause
		\item Résultats attendus \pause
	\end{itemize}
	
\end{frame}
\begin{frame}[plain]
\centering	\Huge \textbf{\textit{MERCI POUR VOTRE ATTENTION}}
\end{frame}
\end{document}
